\documentclass[oneside,12pt,article,a4paper,english,brazil]{abntex2}

% ---
% pacotes e configurações
% ---

\usepackage{lmodern}
\usepackage[T1]{fontenc}
\usepackage[utf8]{inputenc}
\usepackage{indentfirst}
\usepackage{color}
\usepackage{graphicx}
\usepackage{microtype}
\usepackage{tabularx}

% citações
\usepackage[num,overcite]{abntex2cite}
\citebrackets[]
\citeoption{abnt-full-initials=no}

% espaçamento
\setlength{\parindent}{1.3cm}
\setlength{\parskip}{0.2cm}

% ---
% dados
% ---
\titulo{Desenvolvimento de um juiz online e material preparatório para o ensino de programação competitiva}
\autor{Heitor Leite e Daniel Martins}
\local{Contagem}
\data{Maio de 2022}
\instituicao{
  Centro Federal de Educação Tecnológica de Minas Gerais -- CEFET-MG \par
  Departamento de Eletroeletrônica e Computação -- DELCOM \par
  Curso Técnico Integrado em Informática
}
\tipotrabalho{Trabalho de Conclusão de Curso}
\preambulo{Projeto de pesquisa desenvolvido sob a orientação da Professora Elizabeth Duane e coorientação do Professor Alisson Rodrigo dos Santos}
% ---

\definecolor{blue}{RGB}{41,5,195}

% informações do PDF
\makeatletter
\hypersetup{
  pdftitle={\@title}, 
  pdfauthor={\@author},
  pdfsubject={\imprimirpreambulo},
% pdfcreator={LaTeX with abnTeX2},
% pdfkeywords={abnt}{latex}{abntex}{abntex2}{projeto de pesquisa}, 
% colorlinks=true,
% linkcolor=blue,
% citecolor=blue,
% filecolor=magenta,
% urlcolor=blue,
  bookmarksdepth=4
}
\makeatother
% --- 

\makeindex

\begin{document}
\selectlanguage{brazil}
\frenchspacing 

% ---
% elementos pré-textuais
% ---

% capa
\imprimircapa

% folha de rosto
\imprimirfolhaderosto

% lista de ilustrações
%\pdfbookmark[0]{\listfigurename}{lof}
%\listoffigures*
%\cleardoublepage

% lista de tabelas
\pdfbookmark[0]{\listtablename}{lot}
\listoftables*
\cleardoublepage

% lista de abreviaturas e siglas
\begin{siglas}
  \item[OBI] Olimpíada Brasileira de Informática
  \item[CSES] \emph{Code Submission Evaluation System}
  \item[USACO] \emph{USA Computing Olympiad}
  \item[TFC] Torneio Feminino de Computação
  \item[IOI] \emph{International Olympiad in Informatics}
  \item[CIIC] Competição Ibero-Americana de Informática e Computação
  \item[EGOI] \emph{European Girls' Olympiad in Informatics}
  \item[CMS] \emph{Contest Management System}
  \item[JOI] \emph{Japanese Olympiad in Informatics}
  \item[NZOI] \emph{New Zealand Olympiad in Informatics}
  \item[COCI] \emph{Croatian Open Competition in Informatics}
  \item[BOI] \emph{Baltic Olympiad in Informatics}
  \item[CCO] \emph{Canadian Computing Competition}
\end{siglas}

% sumário
\pdfbookmark[0]{\contentsname}{toc}
\tableofcontents*
\cleardoublepage

% ---
% elementos textuais
% ---
\textual

\chapter*{Projeto de pesquisa}
\addcontentsline{toc}{chapter}{PROJETO DE PESQUISA}

\section{Tema geral}
Aprendizado de técnicas e algoritmos utilizados na programação competitiva.

\section{Título da pesquisa}
\thetitle

\subsection{Palavras-chave}
Juiz Online. Programação competitiva. Ensino de programação.

\section{Problema gerador}

A OBI (Olimpíada Brasileira de Informática) é o principal evento de programação competitiva para alunos do ensino médio. Ela se dividide em duas modalidades: iniciação e programação, compostas por provas escritas e de codificação, respectivamente \cite{obi-info}. Na modalidade programação, os estudantes precisam escrever programas que resolvem problemas conhecidos da ciência da computação de maneira eficiente, sendo necessários conhecimentos avançados em algoritmos e estruturas de dados, aritmética, geometria computacional, matemática discreta, grafos, entre outros \cite{obi-ementa} \cite{ioi-syllabus}.

Os conhecimentos exigidos são de difícil acesso para alunos de escolas que não tenham tradição olímpica na OBI. Mesmo em escolas onde há o ensino de programação, não é comum abordar os temas que aparecem nesse tipo de prova. Dessa forma, recursos em português gratuitos e de qualidade sobre programação competitiva a nível do ensino médio são escassos, prova disso é o fato da ementa oficial da OBI citar diversos textos em inglês \cite{obi-ementa}. Existem plataformas como o \href{https://neps.academy}{Neps Academy} com cursos especializados sobre o assunto, mas o acesso ao conteúdo atualizado é pago.

No âmbito internacional, esse problema foi resolvido pelo surgimento de plataformas gratuitas como o \href{https://train.usaco.org/}{USACO Training} \cite{usaco-ioi} e o \href{https://usaco.guide}{USACO Guide}, ambas com conteúdo preparatório para a USACO (\emph{USA Computing Olympiad}), olimpíada de informática estadounidense; e o \href{https://cses.fi/}{CSES} (\emph{Code Submission Evaluation System}), uma coleção de problemas originais de programação competitiva acompanhada de um livro sobre o assunto (\emph{"Competitive Programmer's Handbook"}), desenvolvido pelo programador competitivo finlandês Antti Laaksonen \cite{cses-ioi}.

Nesse contexto, o presente trabalho busca desenvolver uma plataforma de treinamento em programação competitiva gratuita e em português para preparar estudantes para as olimpíadas de informática, em especial as que abrangem os estudantes brasileiros como a OBI, o TFC (Torneio Feminino de Computação), a IOI (\emph{International Olympiad in Informatics}), a CIIC (Competição Ibero-Americana de Informática e Computação) e a EGOI (\emph{European Girls' Olympiad in Informatics}). Tal plataforma conterá um conjunto de problemas composto por problemas originais, bem como problemas traduzidos e arquivados de outras olimpíadas ao redor do mundo, além de uma série de textos introduzindo as técnicas e algoritmos utilizados no mundo da programação competitiva.

\section{Objetivo geral}
Democratizar o acesso à programação competitiva e às olimpíadas de informática. 

\section{Objetivo específico}
Desenvolver uma plataforma contendo um juiz online para problemas de programação competitiva e uma série de textos preparatórios sobre o assunto.

\section{Referencial teórico}
Embasamos nosso projeto nas diferentes plataformas citadas que cumprem um propósito similar fora do escopo nacional \cite{usaco-ioi} \cite{cses-ioi}.

\section{Metodologia}
O desenvolvimento do projeto se dividirá em três partes: a construção do juiz online, do site e do sistema para gerenciamento de textos e a população da plataforma com conteúdo.

Na primeira parte, desenvolveremos um juiz online: um \emph{software} capaz de corrigir soluções para problemas de programação competitiva. O principal desafio nesse tipo de sistema é a segurança, já que é necessário executar código arbitrário fornecido pelo usuário. Para isso, utilizaremos o isolate \cite{isolate-ioi}, uma solução de \emph{sandboxing} desenvolvida exclusivamente para juizes online, sendo utilizada pelo CMS (\emph{Contest Management System}), sistema \emph{open-source} desenvolvido para uso na IOI a partir de 2012 \cite{cms-ioi}. Através do isolate é possível compilar e executar programas enviados pelo usuário de maneira segura, inibindo o acesso a certos recursos do sistema como a leitura e escrita de arquivos e o acesso à rede, além de limitar o tempo de execução e a memória utilizados, parte essencial dos problemas nesse tipo de competição.

No que diz respeito ao desenvolvimento do site, utilizaremos o \emph{framework} Flask da linguagem Python para servir páginas semi-estáticas contendo Javascript mínimo, de modo a enfatizar a velocidade e simplicidade do sistema \cite{drew-flask}. Os textos serão escritos em uma sintaxe costumizada da linguagem Markdown, e compilados para HTML e PDF através do programa Pandoc, com filtros próprios desenvolvidos na linguagem Lua. Optou-se por desenvolver uma sintaxe própria para incorporar elementos que facilitem a redação de problemas e textos sobre programação competitiva, como o suporte nativo à graphviz (um \emph{software} para visualização de grafos) dentro dos documentos.

Por fim, para popular a plataforma com conteúdo, serão desenvolvidos problemas simples que apresentem de maneira direta as principais técnicas usadas. Também serão pesquisados, traduzidos e arquivados no sistema problemas tanto de olimpíadas nacionais, em especial a seletiva para a IOI/CIIC/EGOI (fase final da modalidade programação nível 2 da OBI que seleciona os competidores que representarão o Brasil nas olimpíadas internacionais), cujos problemas passados são de difícil acesso; como de olimpíadas internacionais: IOI, CIIC, EGOI, USACO, JOI (\emph{Japanese Olympiad in Informatics}), NZOI (\emph{New Zealand Olympiad in Informatics}), COCI (\emph{Croatian Open Competition in Informatics}), BOI (\emph{Baltic Olympiad in Informatics}), CCO (\emph{Canadian Computing Competition}), etc. Além disso serão desenvolvidos uma série de textos disponibilizados no site para visualização pelo navegador e compilados para PDF em formato de livro explicando os principais conceitos, algoritmos e técnicas de programação competitiva, como definido pela ementa da OBI, mas não limitado à esta.

\section{Cronograma}

\begin{table}[htb]
  \centering
  \caption{Cronograma de desenvolvimento do projeto}
  \begin{tabularx}{\textwidth}{|c|X|}
    \hline
    \textbf{Meses (2022)} & \textbf{Atividades} \\
    \hline
    Maio a Julho &
    Familiarização com os \emph{frameworks} e \emph{softwares} utilizados \newline
    Construção do juiz online e início da construção do site \\
    \hline
    Agosto e Setembro &
    Integração do sistema do juiz com o sistema do site \newline
    População da plataforma com conteúdo (problemas e textos) \\
    \hline
    Outubro &
    Redação do relatório final \newline
    Finalização do projeto \\
    \hline
    Novembro & Entrega do relatório final e defesa do TCC \\
    \hline
  \end{tabularx}
\end{table}

\phantompart

% ---
% elementos pós-textuais
% ---
\postextual

% referências
\newpage
\bibliography{referencias}

% apêndices
%\begin{apendicesenv}
%\partapendices
%\end{apendicesenv}

% anexos
%\begin{anexosenv}
%\partanexos
%\end{anexosenv}

\phantompart
\printindex
\end{document}
